\section{The HHL Algorithm}
The next section will give a more detailed walkthrough of the HHL algorithm.
Thereby, it will go through all the 5 phases namely, state preperaration, QPE, inversion of eigenvalues, IQPE and lastly the measurement.

\subsection{State Preparation}

In total we have $n_b + n + 1$ qubits. 
In the beginning they are all initialized in their zero state as
\begin{equation}
\begin{split}
\ket{\Psi_0} &= \ket{0\dots0}_b\ \ket{0\dots0}_c\ \ket{0}_a \\
&= \ket{0}_b^{{\otimes n_b}}\ \ket{0}_c^{\otimes b}\ \ket{0}_a 
\end{split}
\end{equation}
We now have to load the vector $\vec{b}$ into the b-register. 
This is achieved by amplitude encoding. 

Todo 

insert formula here

The state $\ket b$ is then loaded into the b-register. Therefore

\begin{equation}
\ket{\Psi_1} = \ket{b}_b\ \ket{0\dots0}_c\ \ket{0}_a
\end{equation}

We have successfully encoded the $\vec{b}$ into our b-register. 
We now continue with the QPE. 

\subsection{Quantum Phase Estimation}
We will only briefly go through the specifics of the QPE and will not discuss each step in detail, as this is not the main topic of this paper. 
For further explanations refer to this paper.
As already mentioned, the QPE is a procedure to evalute an estimate of eigenvalues. 
It consists of three phases, namely the superpositions of the clock-bits via Hadamard gate, controlloed rotation via unitary U and the IQFT.
After QPE we will have an estimate of the eigenvalues of the unitary $U$. 
As we have encode $A$ as as a Hamiltonian $U = e^{iAt}$, the phase of the eigenvalue of U is proportional to the eigenvalue of $A$.
We have to define a scaled version of our eigenvalues $\lambda_j$.
\begin{equation}
\widetilde{\lambda_j} = \frac {N\lambda_jt}{2\pi}
\end{equation}
where $t$ can be choosen freely so that the scaled eigenvalues $\widetilde{\lambda_j}$ are integers.

Thus, the eigenvalues of $A$ will be stored in the c-register after QPE as
\begin{equation}
\begin{split}
\ket{\Psi_2} &= \ket{b}_b \ket{\widetilde{\lambda}}_c\ket{0}_a \\
% &=\left(-\frac{1}{\sqrt{2}} \ket{u_0} \ket{01} +\frac{1}{\sqrt{2}}  \ket{u_1} \ket{10} \right)  \ket{0}_a\\
& =\sum_{j=0}^{N-1} b_j \ket{u_j}_b \ket{\widetilde{\lambda_j}}_c \ket 0_a
\end{split}
\end{equation}
Notice, that the b-register is now a representation of the $\ket b$ state in the eigenbasis $\ket{u_j}$ of $A$.

\subsection{Inversion of the eigenvalues}
In the next step we want to invert the eigenvalues in our state. 
This is achieved by the rotation of the ancilla qubit in the a-register by the eigenvalues in the c-register.
The state of the registers after the rotation looks like this

\begin{equation}
\ket{\Psi_3} = \sum_{j=0}^{N-1} b_j \ket{u_j}_b {\ket{\widetilde{\lambda_j}}}_c \left(\sqrt{1-\frac{C^2}{\widetilde{\lambda_j^2}}}\ket{0}_a + \frac{C}{\widetilde{\lambda_j}} \ket{1}_a\right)
\end{equation}

where $C$ is a constant that should be chosen to be as large as possible to increase the success probability.
Currently, the state of the a-register can either collapse into $\ket 1$ or $\ket 0$. 
The probability of the state collapsing to $\ket 1$ is $\left|\frac C {\widetilde{\lambda_j}}\right|^2$.
If the ancilla bit collapses to $\ket 0$, the whole procedure has to be repeated from the beginning. 
As mentioned earlier this has to be done, because the rotation process is not unitary and has a probability to fail.

We assume that our rotation process was successfull and the ancilla bit collapses to $\ket 1$. 
The registers will look as such
\begin{equation}
\ket{\Psi_4}= \frac {1} {\sqrt{\sum_{j=0}^{N-1}   \left|  \frac{b_jC} {\widetilde{\lambda_j}}\right|^2   }} 
\sum_{j=0}^{N-1} b_j \ket{u_j}_b \ket{\widetilde{\lambda_j}}_c \frac{C}{\widetilde{\lambda_j}} \ket1_a
\end{equation}
Note that term in front of the sum is just a factor to normalize the state. 
Lets call this normalization factor $D$.
We see that we now have a term $\frac{C}{\widetilde{\lambda_j}}$ that represents the inverted eigenvalues. 
This term can be moved around freely as it is just a scalar and can be moved, such that it is applied to the b-register
\begin{equation}
\ket{\Psi_4}= D
\sum_{j=0}^{N-1} \frac{C}{\widetilde{\lambda_j}} b_j \ket{u_j}_b \ket{\widetilde{\lambda_j}}_c \ket1_a
\end{equation}

If we go back to our idea from the mathematical overview section, our b-register is in the same form as our solution state
\begin{equation}
\ket{x} = \sum_{i=0}^{N-1} \lambda_i^{-1} b_j\ket{u_j}\\
\end{equation}

That means our solution is already encoded in our registers. 
The problem here is, that we cannot read the solution out yet.
This has to with the states in the b-register and c-register being entangled with each other. 
That means we cannot convert the b-register into a $\ket0 / \ket 1$ measurement.
In the following we have to undo all operations to unentangle the state in the b-register and c-register, to achieve the correct result.

\subsection{Inverse Quantum Phase estimation.}
The unentangling of the registers is achieved throught the IQPE which just backtracks all calculations of the QPE.
We are left with the following state

\begin{equation}
\begin{split}
\ket{\Psi_5} &= \frac {1} {\sqrt{\sum_{j=0}^{N-1} \left| \frac{b_jC} {\lambda_j}\right|^2}}
\sum_{j=0}^{N-1}\frac{ C}{\lambda_j}   b_j \ket{u_j}_b \ket{0}_c^{\otimes n} \ket{1}_a\\
&= \frac {C} {\sqrt{\sum_{j=0}^{2^{n_b}-1} \left| \frac{b_jC} {\lambda_j}\right|^2}}
\ket{x}_b \ket{0}^{\otimes n}_c \ket{1}_a
\end{split}
\end{equation}

The a-register is untouched and still hold the $\ket 1$ state as before. 
The c-register however is reset to the zero state $\ket{0}^{\otimes n}_c$ as in the beginning of the process. 
It is now unentangled from the b-register.
The b-register now contains the solution $\ket x$ after the uncomputation and can be measuremed correctly.

We can furthermore simplify the term as we assume that  $C$ is real and that the eigenvectors $\ket{u_i}$, and the $\ket b$ state are normalized.
Then we can simplify to 

\begin{equation}
\begin{split}
\ket{\Psi_5} &= \frac {1} {\sqrt{\sum_{j=0}^{N-1} \left| \frac{b_j} {\lambda_j}\right|^2}}
\ket{x}_b \ket{0}^{\otimes n}_c \ket{1}_a\\
&= \ket{x}_b \ket{0}^{\otimes n}_c \ket{1}_a
\end{split}
\end{equation}

We can now  read out the result $\ket x$ in the b-register.

\subsection{Measurement}
As mentioned earlier, we cannot obtain the whole solution for the $\vec x$ as reading out all the entries would cost us $\mathcal{O}(N)$ steps.
This would ommit our speedup of $\mathcal{O}(log N)$. 
That means that by measuring, we will only obtain an estimate of specific features of $\ket x$.
Using a linear operator $M$ we can perform various measurements on $\ket x$ by calculating the inner product as such
\begin{equation}
    \bra x M \ket x
\end{equation}
With this we can extract various statistical features of $\ket x$ like the norm of the vector, the average of the weight of the components, moments, probability distributions, localization and concentration in specific regions, etc.


























