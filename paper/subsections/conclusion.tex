
\section{Conclusion}

In conclusion, the HHL algorithm offers a significant improvement over its classical counterpart, considering its constraints.
Sparsity, a low condition number and error are needed for the HHL algorithm to work efficiently.
Additionally, ways to load the input vector faster, for example with Qram, are needed. 
In the end, we observed, that the sparsity and conditioning of the problem matrices $A$, have a comparable dependency on the runtime, but the error dependency is exponentially worse. 

Despite these constraints, the HHL algorithm has made significant contributions, especially in the field of quantum machine learning. 
Since there are no ground-breaking applications so far, the Shor's algorithm, that has the ability to break the RSA encryption, further use cases need to be found.
Nonetheless, the ongoing discussions and research on the algorithm enable us to discover new optimizations and applications.
The advances in quantum algorithms, including the HHL algorithm, have laid the foundation for future developments in the field.
Because of the generality of the HHL algorithm, it has the potential to be used as a subroutine in various domains.

While experiments have demonstrated that the HHL algorithm works on small quantum computers with high fidelity, 
more advancements, as better error correction and scaling are needed to unlock the full potential of quantum computing.



